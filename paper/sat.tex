\documentclass{sig-alternate}
\usepackage{listings}
\usepackage{amsfonts}
\begin{document}
\conferenceinfo{USNA}{Annapolis, MD, USA}

\title{Examination of SMT to SAT Reduction}

\numberofauthors{1}
\author{
\alignauthor
MIDN R Blair Mason, USN\\
   \affaddr{1 Wilson Rd. \#13623}\\
   \affaddr{Annapolis, MD 21412}\\
   \email{m164122@usna.edu}
}
\date{12 December 2014}

\maketitle

\begin{abstract}
This paper is an overview of current techniques for reduction of
SMT problems over the \texttt{QF\_BV} theory to boolean SAT instances.
It compares the performance of a basic eager implementation, which
takes input in C via a clang plugin, to the Z3 SMT solver.  It then
compares the performance of these to two different SAT backends,
minisat and cryptominisat.

This experiment found that the Z3 had a significant performance
improvement (~5.5x) over the basic eager implementation in factoring
40-bit semiprimes, but this improvement was mitigated somewhat (down
to 1.12x) with the use of an optimized SAT backend.
\end{abstract}

\section{Introduction}
Satisfiability modulo theories (SMT) solvers are commonly used in many
domains for correctness verification.  However, they are broadly useful
as the NP-completeness of the satisfiability problem implies that any
problem in NP can be transformed into an equivalent satisfiability
problem.

One of the most useful theories in SMT for reverse engineering purposes
is the theory of quantifier free fixed-length bitvectors, 
\texttt{QF\_BV}.  Many programs written in low level languages are in
essence descriptions of mathematical transformations over finite length
bit vectors, so these programs are easily transformed into an appropriate
representation in \texttt{QF\_BV}.  Computations performed by these
programs can have important security or correctness implications, and
SMT solvers can perform a range of functions from proving the safety of
memory accesses to computing function preimages.

\subsection{Transformation into SMT}

Simple functions in C are very easy to transform into an equivalent
problem in \texttt{QF\_BV}.  For example, the following C function
takes two 32-bit bitvectors (assuming a platform with 32-bit int) and
returns the bitwise exclusive or of the two bitvectors and a constant
(0xdeadbeef):

\begin{lstlisting}[language=C]
unsigned foo(unsigned bar, unsigned baz) {
    bar ^= 0xdeadbeef;
    bar |= baz;
    return bar;
}
\end{lstlisting}

This can be encoded in the SMT-Lib 2.0 format as:

\begin{lstlisting}
(define-fun foo
    ((bar (_ BitVec 32)) (baz (_ BitVec 32)))
    (_ BitVec 32)
    (bvxor (bvxor bar #xdeadbeef) baz))
\end{lstlisting}

\subsection{Challenges in applying SMT to C}
Arrays, structures, and unions can all be similarly handled in the
\texttt{QF\_BV} theory.  The difficulty lies in handling pointers,
input, and loops.  SMT solvers can, in some cases, be used to prove
the validity of arbitrary assertions, including the validity of
memory accesses, such as in \cite{Gries07}.  However, pointer
arithmetic and looping make this task significantly more difficult.
Determining satisfiability of a program that contains a loop is
reducible to the halting problem in the general case.  Only loops
that can be statically unwound can be transformed into an
equivalent SMT representation.

One approach to solving this problem is to guess an upper bound for
the number of times a loop may execute and add logic that will poison
the output if the loop executes additional times.  This can be done
in an iterative fashion, increasing the upper bound with each run,
until the solver is able to either determine the actual upper bound
on the loop's execution or find a model that satisfies the constraints
given to the solver.  Since this upper bound can be arbitrarily large,
the solver cannot prove the satisfiability of the given model in the
general case.

Input can be handled as simply unknown bitvectors.  Interactive or
hardware input that comes in as part of an unbounded loop is difficult
or impossible to analyze, but if the amount of input to the program
is finite it can be modeled as unknown bits in the \texttt{QF\_BV}
representation.

\subsection{Transformation to SAT}

SMT Solvers, given a problem in \texttt{QF\_BV}, must then perform
further processing on the problem in order to solve it.  Modern SMT
solvers perform various operations on the SMT problem, such as
propagating constants and simplifying operations at the bitvector
level.  Early SAT solvers, however, would immediately transform the
problem into a \textsc{Circuit-SAT} instance.  This is referred to
as the \textit{eager} approach.  The transformation from an SMT
representation to a SAT representation loses information about the
semantic relationships between data, so "obvious" facts in SMT
(e.g. $a + b = b + a$) are much more difficult for the SAT solver
to learn.  In practice, modern SMT solvers communicate with the
underlying SAT solver to eliminate duplicate search paths.

After any transformations at the SMT level are applied, most SMT
solvers convert the \texttt{QF\_BV} problem representation into a
\textsc{Circuit-SAT} instance through a process called
\textit{bit-blasting}.  This process is essentially the same as
recreating the circuitry that a processor would use to perform
the computation using abstract logic gates.

Addition and subtraction of bitvectors can be implemented with a
series of one bit full adder circuits.  Shifting by a constant
amount is trivial to implement.  Shifting by variable amounts can
be implemented with barrel shifters.

A conditional statement \verb|b ? t : f| for the bit b and bitvectors
t and f can be emulated by the computation 
$((b \wedge t_i) \vee (\neg b \wedge f_i) \forall t_i \in t, f_i \in f)$.
With conditional expressions, one can implement multiplication as a
series of shifts and conditional adds.  However, this is the binary
equivalent of standard $O(n^2)$ multiplication.  A full adder uses five
gates to perform the addition and carry computation.  The shifts can be
done without adding any gates, but the logic to make the add conditional
requires an additional three gates per bit, for a total of eight gates
per bit per add.  So, performing a multiplication of two 64-bit integers
and preserving the entire 128-bit result requires 131,072 gates using
the naive algorithm.

Because multiplication and division are nonlinear operations on bitvectors,
these operations are significantly more challenging for SMT solvers to
solve.

\subsection{Transformation to CNF}

The \textsc{Circuit-SAT} problem that encodes the \texttt{QF\_BV}
problem is then transformed once again into a CNF-SAT problem.  This
process is a simple application of the Tseitin transformation from
\cite{tseitin}.  In this process, the solver creates a list of clauses
and visits each gate in the circuit, appending clauses to the list
as it goes.  Each gate adds three clauses to the output CNF
representation.  Though this is usually very far from the minimal CNF
representation of the circuit, obtaining such a minimal CNF using
known algorithms requires exponential time.  The resultant CNF
description is logically equivalent to the original circuit, and any
model that satisfies the CNF also satisfies the original circuit
after discarding intermediate variables.

\subsection{Solving a CNF-SAT problem}

CNF-SAT problems are usually solved using the DPLL algorithm described
in \cite{dpll}.  This algorithm is a simple backtracking depth first
search with two added steps at each state:

\begin{enumerate}
    \item For all clauses containing only one variable $V$, remove
        all instances of $\neg V$ and remove all clauses containing
        $V$ (and analogously for clauses containing only $\neg V$).
    \item If any variable $V$ occurs either always as $V$ or always
        as $\neg V$, set it to true or false, respectively, and
        remove all clauses with this variable.
\end{enumerate}

At any point, the current state is unsatisfiable if the CNF contains
empty clauses, and the current state is a model that satisfies the
problem if the current state is empty.

For performance, the critical step in the algorithm is deciding which
variable to assign next.  This is where a modern SMT solvers use their
additional information to remove large amounts of the search space and
prioritize important variables.

\section{Performance Comparisons}

For this paper, a clang plugin was developed that, given a function in
a subset of C and a desired return value, created an equivalent SAT
problem instance.  It could then either dump the instance to a DIMACS
file or solve it using the default \texttt{solve()} function in
\texttt{libminisat.so}.  This plugin was then compared with Z3, a
highly optimized SMT solver from Microsoft Research (\cite{demoura}),
and the minisat and cryptominisat demo solvers, both of which are
optimized over the default solvers in the library.

\subsection{Problem Description}

The problem used in this experiment was a simple factoring problem.
The problem description was:

\medskip

\noindent \hangindent=0.7cm \textbf{Given:} $n \in \{ 0, 1 \}^k$
for some $k \in \mathbb{Z}$

\noindent \hangindent=0.7cm \textbf{Find:} $p, q \in \{ 0, 1 \}
^{\frac k 2} \ni 0^{\frac k 2}p \times 0^{\frac k 2}q = n$

\medskip

This problem was chosen as the nonlinear nature of multiplication
makes it easy to generate SMT problems with significant runtimes.
For many simple linear problems, parsing the problem description
comprises a non-trivial portion of the total runtime.  However,
this approach may be unfair to the Z3, as the information lost
transforming multiplication into SAT (such as commutativity at
the bitvector level) is much less than the information lost for
many linear operations, such as addition (commutativity for each
bit). Also, since multiplication is so inefficient to solve using
a SMT solver, users generally attempt to avoid multiplications if
possible, so these numbers may not be representative of general
usage patterns.

\section{Factoring 32-bit semiprimes}

The first test was conducted with 32-bit semiprimes.  The SMT-Lib
input to Z3 is given in listing \ref{lst:smtlib}, for various values of
n (0x5ddddbef is just an example).

\begin{lstlisting}[language=Lisp,caption={SMT-Lib Input},label={lst:smtlib}]
(declare-const p (_ BitVec 16))
(declare-const q (_ BitVec 16))

(define-fun extend ((x (_ BitVec 16))) (_ BitVec 32)
    (concat #x0000 x))

(assert (=
    (bvmul (extend p) (extend q))
    #x5ddddbef))

(check-sat)
(get-model)
\end{lstlisting}

The equivalent C code is given in listing \ref{lst:c_input}.  For the C solver the
desired return value was given as input to the solver program.

\begin{lstlisting}[language=C,caption={C Language input},label={lst:c_input}]
#include <stdint.h>
uint32_t mul(uint16_t p_, uint16_t q_) {
    uint32_t p(p_), q(q_);
    return p*q;
}
\end{lstlisting}

A comparison of the performance of Z3 with cxxsat (the custom clang plugin) is
shown in figure \ref{fig:bit16}, along with the 95\% confidence intervals for
each measurement.  A new semiprime was generated for each test run, with the same
semiprime being given to each solver in each run.  The Z3 solver exhibited a
speedup of 5.58 over cxxsat in this test case, but both solvers gave results
in less than 10 seconds.

\begin{figure}
    \input{bit16}
    \caption{Time to factor 32-bit semiprimes}
    \label{fig:bit16}
\end{figure}

\subsection{Factoring 40-bit semiprimes (64-bit math)}
The solvers were then run with 40-bit semiprimes, but the numbers were given
to the solvers as 64-bit numbers, in order to maintain the compatibility with
C source code and to test the ability of the solvers to eliminate unnecessary
information.  These results are displayed in figure \ref{fig:bit20}.  Only
eight measurements were taken using 64-bit math due to the long average
runtimes of these tests.  In addition, the ninth run had to be canceled as
cxxsat had occupied 100\% of CPU time for over two hours and failed to produce
a result.  Z3 exhibited a speedup of 8.71 over cxxsat for this benchmark, which
is 1.56 times larger than the speedup for 32-bit math.  This indicates that
the SMT solver is better at eliminating the unnecessary bits of information
than the SAT solver backend (minisat) of cxxsat.

\begin{figure}
    \input{bit20}
    \caption{Time to factor 40-bit semiprimes using 64-bit math}
    \label{fig:bit20}
\end{figure}

\subsection{Factoring 40-bit semiprimes (40-bit math)}

One last benchmark was conducted for this experiment.  Using the transformation
library created for cxxsat, a custom program was written that allows the user
to select a number of bits for the factoring problem rather than be constrained
to C data types.  Since this program no longer has to parse an input description,
its times are slightly lower than a general purpose solver would have.  However,
the parse time spent by cxxsat, including problem generation, was less than 0.2
seconds for these inputs, so the data are not significantly skewed.

This program was then compared with Z3.  It also was instructed to dump the
problem descriptions to a DIMACS file.  The minisat and cryptominisat SAT solvers
were then run on this problem description file.  minisat is the same benchmark
used as the backend of cxxsat, so it serves as a comparison between a generic
SAT solving approach and a performance tuned approach using the same engine.
The comparison with cryptominisat shows the difference between different
highly efficient solvers.  These measurements are shown in figure
\ref{fig:solver_comparison}.  Speedups with reference to cxxsat are shown in
table \ref{tab:solver_comparison}.

\begin{figure}
    \input{results}
    \caption{Time to factor 40-bit semiprimes}
    \label{fig:solver_comparison}
\end{figure}

\begin{table}
    \caption{Speedups (40-bits, to cxxsat)}
    \label{tab:solver_comparison}
    \begin{center}
        \smallskip
        \begin{tabular}[]{c | c}
            solver        & speedup \\
            \hline
            Z3            & 5.51 \\
            cxxsat        & 1.00 \\
            minisat       & 2.17 \\
            cryptominisat & 4.89

        \end{tabular}
    \end{center}
\end{table}

The speedup of Z3 of 5.51 is very similar to the 5.58 speedup observed with
32-bit math.  This shows that the optimizations applied at the SMT level by
Z3 are (in this case) roughly constant factor optimizations.  The advantages
of these optimizations over the simple eager approach do not grow or shrink
significantly with changes in the size of the problem.

These results also highlight the importance of properly tuning the SAT solver
backend.  Significant gains were made using the high-performance demo solvers
available with minisat and cryptominisat.  Though the DPLL algorithm is
fairly simple, optimizations as simple as educated variable selection can lead
to obtaining a model significantly faster.  cryptominisat also includes several
optimizations that lead to large performance gains, including XOR clause
recovery from CNF constraints \cite{soos}.

The speedup of Z3 relative to cryptominisat was only 1.13, showing that
a sufficiently sophisticated SAT solver can achieve comparable performance
to a SMT solver on problems, even when using the eager approach.  One must,
however, note the limited amount of information discarded when bit-blasting
a multiplication operation in comparison to typical linear operations such
as bitwise operations and addition/subtraction.  Therefore, this claim may
not be relevant for other (not measured) classes of problems.

\subsection{Comparing 40-bit and 64-bit results}

Solving the problem using 64-bit math took Z3 17.1 times as long as solving
the problem using 40-bit math.  cxxsat fared even worse, taking 27.2 times
as long.  This significant increase in runtime is due to the choice of
multiplication in the problem description, which produces a SAT problem that
is nonlinear ($O(n^2)$ using the basic algorithm) in size with respect to the
size of the input.  However, Z3 did not suffer as much due to its ability to
reason about the multiplication at the word level rather than the clause
level (though it still exhibited a significant decrease in performance).

\section{Conclusions}
The eager approach is still a viable algorithm for proving various properties
of computer programs or generating function preimages for reverse engineering.
SMT solvers are capable of applying optimizations that result in significant
performance gains over a basic eager approach.  However, use of a well-designed
intelligent SAT solver can compensate for much of the information discarded when
transitioning from a SMT problem representation to a SAT problem representation.
For a generic semiprime factoring problem, an SMT solver exhibited a speedup of
5.51 over a completely naive approach.  However, use of a high performance SAT
solver in the backend reduced this speedup to only 1.12.  These results must
be qualified by stating that they were measured over nonlinear (multiplicative)
arithmetic, and thus they may not generalize well to many of the classes of
problems in which SAT solvers are currently used.

\section{Acknowledgments}
The author would like to thank the clang developers for their work in producing
a hackable open source c++ parser.  He also wishes to thank the U.S. department
of defense for considering it a good use of their money to educate him, and Dr.
Aviv for creating the \textit{SuperSecretCyber} elective and accepting (sort of)
his constant excuses for why all his work is turned in at the last minute.

\bibliographystyle{abbrv}
\bibliography{sat}
\end{document}
